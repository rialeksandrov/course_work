\chapter*{Алгоритм генетического поиска для решения задачи планирования вычисления композитных приложений в распределенной вычислительной среде}
\addcontentsline{toc}{chapter}{2. Алгоритм генетического поиска для решения задачи планирования вычисления композитных приложений в распределенной вычислительной среде}
\begin{spacing}{1.3}

Генетический алгоритм поиска является универсальной концепцией для приближенного решения трудновычислимых задач. Его приложение к задаче планирования описано более подробно здесь \cite{GA}. Здесь же будет краткое описание работы алгоритма.

Общий вид генетического алгоритма одинаков для всех приложений:

\begin{algorithmic}
\STATE \textit{1. Инициализация начального поколения}
    \WHILE{Критерий остановки не выполнен}
    	\STATE \textit{2. Кроссовер} 
    	\STATE \textit{3. Мутация}
    	\STATE \textit{4. Оценка поколения}
    	\STATE \textit{5. Выбор в новое поколение}
    \ENDWHILE
\end{algorithmic}

Основную роль в генетическом алгоритме играет элемент поколения --- хромосома. В рамках задачи планирования хромосома выглядит как пара строк. Одна строка, будем называть его $mat$ задает соответствие между задачей и ресурсом. Вторая строка --- $ss$ задает последовательность задач в которой они будут запланированы. Важно что $ss$ является некоторой топологической сортировкой ацикличного графа композитного приложения.

В начальное поколение генетического алгоритма нужно добавлять совсем случайные хромосомы и хромосомы которые соответствуют решениям построенным другими различными эвристиками, например с помощью DLS или HEFT. Тогда генетический алгоритмы может улучшить какое-то из решений с помощью случайных процессов которые происходят с хромосомами. 


Кроссовер является одним из наиболее важных частей алгоритма. Он помогает обмениваться информацией между двумя хромосомами --- создавать новые хромосомы отличные от родительских. Кроссовер между двумя $mat$ строками просто режет их в одинаковых частях и меняет отрезанные части. Кроссовер между двумя $ss$  строками также режет их в одинаковых местах и потом в отрезанной части строки задачи переставляются в порядке задаваемым строкой с которой происходит кроссовер. 

Мутация позволяет создавать хромосомы с совсем новым решением. Процесс мутации для $mat$ строки очень простой --- выбирается случайная задача и ей в соответствии задается случайный ресурс. Мутация же $ss$ строки происходит чуть сложнее. Выбирается случайная задача и переносится в случайное место такое, чтобы новая строка также будет являться топологической сортировкой.



Оценка хромосомы осуществляется с помощью симуляции или того расписания которое было построено (оно дает приблизительное время выполнения композитного приложения так как в составлении расписания не учитывается общее пользование сети). После того как все элементы поколения получили оценку, они сортируются на основе этой оценки и потом в зависимости от места в порядке сортировки им присуждается вероятность с которой они будут выбраны. Выбор производится с повторением, то есть некоторые хромосомы могут дублироваться, однако это будут скорее всего хромосомы с хорошим расписанием. Для $i$-го элемента будет такая вероятность: 
$$
P_i = R^{N - i - 1} (1 - R) / (1 - R^N)
$$
$N$ --- размер популяции, $R$ - параметр алгоритма.





\end{spacing}