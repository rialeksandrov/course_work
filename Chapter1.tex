\chapter*{Постановка задачи}
\addcontentsline{toc}{chapter}{Постановка задачи}
\begin{spacing}{1.4}

Введем две модели - модель вычислительной среды и модель вычислительного приложения, которая как раз описывается с помощью DAG (Directed acyclic graph). Это две важные части задачи планирования. Также нужно отметить, что предполагается наличие некоторой управляющей системы которая будет распределять задачи на вычислительные ресурсы и запускать их в нужное время. Будем считать что всевозможные накладные расходы работ этой части системы малы и ими можно пренебречь.

Вычислительная среда состоит из ресурсов $R_i~(1 \leq i \leq M)$ с различной производительностью. Между каждой парой ресурсов есть сетевое соединение с заданной пропускной способностью. При этом эффектами взаимодействия потоков данных обычно пренебрегают. Возможен учет дополнительных факторов и характеристик: объем доступной памяти, количество ядер процессора или вероятность отказа ресурса. В дальнейшем будем предполагать что вероятность отказа за время работы отдельного приложения крайне мала и может игнорироваться.

Композитные приложения (КП) удобно описывать на языке потоков работ (workflow). Поток работ обычно представляется в виде направленного ациклического графа (DAG). Вершины графа $T_a~(1 \leq a \leq N) $ отражают  выполнение отдельной задачи в КП, а ребра $C_{ab}$ задают зависимость или связь между задачами. Кроме того на ребрах определен объем данных $c_{ab}$ который нужно передать от задачи $T_a$  к задаче $T_b$ только после завершения задачи $T_a$ и до начала выполнения задачи $T_b$.  Если присутствует связь $C_{ab}$ значит задачу  $T_a$ называют (непосредственным) родителем задачи  $T_b$ , а задачу $T_b$ дочерней   $T_a$. Дочерние задачи не могут быть начаты пока не завершились все их родители. Для всех рассматриваемых далее методов планирования необходима возможность оценить время выполнения задач на любом ресурсе вычислительной среды. Обозначим эту оценку как $EET(T_a, R_i)$ $(EET = \text{Estimated Execution Time})$ и будем считать её известной. Аналогично времени выполнения $EET$, обозначим время передачи данных от задачи $T_a$ к $T_b$ как $ECOMT(C_{ab}, R_i, R_j)$ $(ECOMT = \text{Estimated Communication Time})$. Есть различные подходы к оценке $ECOMT$, но чаще всего используют простую линейную модель $$ECOMT (c_{ab}, R_i, R_j) = t_{con} + \frac{c_{ab}}{L_{ij}}$$
,где $t_{con}$ латентность соединения, $L_{ij}$ - пропускная способность сети между ресурсами $R_i$, $R_j$. 

Отметим, что при практической реализации планирования вопрос вычисления $EET$ и оценки $ECOMT_{ab}$ является сложной независимой задачей.



В рамках введенной модели мы можем описать время завершения каждой задачи и всего приложения в целом. Пусть задача $T_a$ выполняется на ресурсе $R_j$, а ее родительские задачи $T_b$ - на ресурсах $R_{j_b}$. Тогда:
$$
ECT(T_a) = EET(T_a, R_i) + \max_{b \in \text{родительские задачи}} (ECOMT(c_{ab}, R_i, R_{j_b}) + ECT (T_b, R_{j_b})) 
$$
Оценка времени завершения последней задачи в композитном приложении $ECT_{\max} = \max_a (ECT(T_a)$ определяет время выполнения всего композитного приложения. 

В данной работе будут рассмотрены алгоритмы планирования с наилучшим качеством, то есть алгоритмы которые минимизируют время выполнения всего  приложения.




\end{spacing}