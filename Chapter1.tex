\chapter*{Постановка задачи}
\addcontentsline{toc}{chapter}{Постановка задачи}
\begin{spacing}{1.4}

Для постановки задачи планирования необходимо ввести две модели - модель вычислительной среды и модель вычислительного приложения, которая как раз описывается с помощью DAG. Кроме того, неявно предполагается наличие управляющей системы или брокера выполнения, то есть программного комплекса, непосредственно запускающего задачи. Однако с точки зрения задачи планирования достаточно предположить что накладные расходы управляющей системы пренебрежимо малы.

Вычислительная среда состоит из ресурсов $R_i~(1 \leq i \leq M)$ с различной производительностью. Между Каждой парой ресурсов есть сетевое соединение с заданной пропускной способностью. При этом эффектами взаимодействия потоков данных обычно пренебрегают. Возможен учет дополнительных факторов и характеристик: объем доступной памяти, количество ядер процессора или вероятность отказа ресурса. В дальнейшем будем предполагать что вероятность отказа  за время работы отдельного приложения крайне мала и может игнорироваться.


Композитные приложения (КП) удобно описывать на языке потоков работ (workflow). Поток работ обычно представляется в виде направленного ациклического графа (DAG). Вершины графа $T_a~(1 \leq a \leq N) $ отражают  выполнение отдельной задачи в КП, а ребра $C_{ab}$ (количество ребер $|C_{ab}| = E$ задают зависимость или связь между задачами. Если присутствует связь $C_{ab}$ значит задачу  $T_a$ называют (непосредственным) родителем задачи  $T_b$ , а задачу $T_b$ дочерней   $T_a$. Дочерние задачи не могут быть начаты пока не завершились все их родители. Для всех рассматриваемых далее методов планирования необходима возможность оценить время выполнения задач на любом ресурсе вычислительной среды. Обозначим эту оценку как $EET(T_a, R_i)$ $(EET = \text{Estimated Execution Time})$ и будем считать её известной. 

Для каждой связи $C_{ab}$ определен объем данных $c_{ab}$ передаваемых от родительской задачи к дочерней. Аналогично времени выполнения $EET$, обозначим время передачи данных от задачи $T_a$ к $T_b$ как $ECOMT(C_{ab}, R_i, R_j)$ $(ECOMT = \text{Estimated Communication Time})$. Есть различные подходы к оценке $ECOMT$, но чаще всего используют простую линейную модель $$ECOMT (c_{ab}, R_i, R_j) = t_{con} + \frac{c_{ab}}{L_{ij}}$$
,где $t_{con}$ латентность соединения, $L_{ij}$ - пропускная способность сети между ресурсами $R_i$, $R_j$. 

Отметим, что при практической реализации планирования вопрос вычисления $EET$ и оценки $ECOMT_{ab}$ является сложной независимой задачей.



В рамках введенной модели мы можем описать время завершения каждой задачи и всего приложения в целом. Пусть задача $T_a$ выполняется на ресурсе $R_j$, а ее родительские задачи $T_b$ - на ресурсах $R_{j_b}$. Тогда:
$$
ECT(T_a) = EET(T_a, R_i) + \max_{b \in \text{родительские задачи}} (ECOMT(c_{ab}, R_i, R_{j_b}) + ECT (T_b, R_{j_b})) 
$$
Оценка времени завершения последней задачи в композитном приложении $ECT_{\max} = \max_a (ECT(T_a)$ определяет время выполнения всего композитного приложения. 

В данной работе будут рассмотрены алгоритма планирования с наилучшем качеством, то есть минимизации времени выполнения.




\end{spacing}