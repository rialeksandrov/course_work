\chapter*{Введение}
\addcontentsline{toc}{chapter}{Введение}


%{\center \chapter {Введение} }

%{\center \section{Введение} }

\begin{spacing}{1.3}
В работе представлена задача планирования вычисления DAG в распределенных системах вычисления. Согласно статье \cite{NPCOMP}, задача поиска оптимального расписания является NP-полной. Поэтому для наиболее оптимального и при этом быстрого решения данной задачи было придумано много различных приближенных методов, большое количество которых описано в статье-обзоре \cite{NZOV}.  В главе 1 подробно рассмотрена постановка задачи. В главе 2 описаны классические эвристические алгоритмы планирования. В главе 3 представлено применение генетического алгоритма к данной задаче. В главе 4 приведены результаты сравнения алгоритмов затронутых в данной работе.
 
\end{spacing}

