\chapter*{Введение}
\addcontentsline{toc}{chapter}{Введение}


%{\center \chapter {Введение} }

%{\center \section{Введение} }

\begin{spacing}{1.3}
В работе представлена задача планирования вычисления DAG в распределенных вычислительных системах. Согласно статье \cite{NPCOMP}, задача поиска оптимального расписания является NP-полной. Поэтому для наиболее качественного и при этом быстрого решения данной задачи было придумано много различных приближенных методов. Часть из них будут затронуты в данной работе --- классические алгоритмы решения этой задачи, а также применение генетического алгоритма. Кроме того будет проведено сравнение данных подходов. В работе представлена постановка задачи. В главе 1 описаны классические эвристические алгоритмы планирования. В главе 2 представлена адаптация генетического алгоритма к данной задаче. В главе 3 приведены результаты проведенных экспериментов и сравнение алгоритмов затронутых в данной работе.
 
\end{spacing}

