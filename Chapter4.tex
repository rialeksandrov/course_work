\chapter*{Эксперименты}
\addcontentsline{toc}{chapter}{3. Эксперименты}
\begin{spacing}{1.3}

В рамках этой работы были реализованы алгоритмы DLS и генетический алгоритм.  Реализации алгоритмов добавлены в открытый сборник алгоритмов, который находится на сервисе \href{https://github.com/alexmnazarenko/pysimgrid}{github} \cite{GIT}. Данный репозиторий содержит алгоритмы, тестовые приложения и различные конфигурации вычислительных систем и самое важное там находится удобная обертка над программой моделирования вычисления приложения в распределенной системе.

Ниже представлены результаты работы алгоритмов на разных конфигурациях вычислительной сети и разных композитных приложениях. Все результаты нормированы по значению работы алгоритма HEFT. Видно что DLS несильно отличается от него. GA в свою очередь улучшает результат работы стандартных методов с помощью мутаций и кроссовера. Параметры использованные для получения таких результатов следующие: $N = 50$, $Iterations = 300$, $P_{mutation} = 1.0$, $P_{crossover} = 0.8$.

~

\begin{center}
\begin{tabular}{|c|c|c|c || c|c|c|c|}
\hline
Nets & HEFT &  DLS &  GA & Nets & HEFT &  DLS &  GA \\ 
\hline
\multicolumn{4}{|c||}{CyberShake} & \multicolumn{4}{c|}{ Inspiral} \\ 
\hline
5     &     1.0000 &     1.0008 &     0.9892  & 5 &     1.0000 &     0.9994 &     0.9740 \\
\hline
10              &     1.0000 &     1.0014 &     0.9753  & 10              &     1.0000 &     1.0029 &     0.9599 \\
\hline
20              &     1.0000 &     0.9993 &     0.9833 & 20              &     1.0000 &     1.0107 &     0.9538 \\
\hline
\multicolumn{4}{|c||}{Epigenomics} & \multicolumn{4}{c|}{Montage} \\ 
\hline
5               &     1.0000 &     1.0028 &     0.9680  & 5               &     1.0000 &     0.9989 &     0.9905 \\
\hline
10              &     1.0000 &     1.0090 &     0.9500 & 10              &     1.0000 &     0.9927 &     0.9855 \\
\hline
20              &     1.0000 &     1.0010 &     0.9552 & 20              &     1.0000 &     0.9759 &     0.9697 \\
\hline

\end{tabular}
\end{center}


\end{spacing}